% !TeX root = ./main.tex

% 论文基本信息配置

\sysusetup{
  %******************************
  % 注意:
  %   1. 配置里面不要出现空行
  %   2. 不需要的配置信息可以删除
  %   3. 建议先阅读文档中所有关于选项的说明
  %******************************
  %
  output = print,
  %
  % 标题
  %   可使用“\\”命令手动控制换行
  %
  title  = {基于罗刹驭空的三人反击配队},
  title*  = {英文标题(sdu应该不需要)},
  %
  % 学位
  % 硕士 或 博士
  % degree-name  = {硕士},
  %
  % 培养单位
  %   填写所属院系的全名
  %
  % department = {计算机学院},
  %
  % 工程硕士的名称应为 "工程硕士 [专业名称]"
  % discipline  = {[专业名称]},
  %
  % 姓名
  author  = {克拉拉},
  id = {114514},
  college = {教令院},
  major = {机器人工学},
  grade = {1919810级},
  %
  % 指导教师
  supervisor  = {史瓦罗},
  % supervisor-title  = {[职称A]},
  % jointsupervisor = {[姓名B]},
  % jointsupervisor-title  = {[职称B]},
  %
  % 答辩委员会成员
  committee-chair = {},
  committee-member1 = {},
  committee-member2 = {},
  committee-member3 = {},
  committee-member4 = {},
  committee-member5 = {},
  % 日期
  %   使用 ISO 格式;默认为当前时间
  %
  % date = {2020-12-01},
  %
  %
  % 密级: 公开
  secret-level = {公开},
  % secret-year  = {10},
  %
  % 
  series-id = {[学号]},
}

% 载入所需的宏包

% 可以使用 nomencl 生成符号和缩略语说明
% \usepackage{nomencl}
% \makenomenclature

% 表格加脚注
\usepackage{threeparttable}

% 表格中支持跨行
\usepackage{multirow}

% 固定宽度的表格。放在 hyperref 之前的话,tabularx 里的 footnote 显示不出来。
% \usepackage{tabularx}

% 跨页表格
% \usepackage{longtable}

\usepackage{multicol}

% 量和单位
\usepackage{siunitx}

% 定理类环境宏包
\usepackage{amsthm}
% 也可以使用 ntheorem
% \usepackage[amsmath,thmmarks,hyperref]{ntheorem}

% 参考文献使用 BibTeX + natbib 宏包
% 顺序编码制
\usepackage[sort]{natbib}
\bibliographystyle{sysuthesis-numeric}

% 著者-出版年制
% \usepackage{natbib}
% \bibliographystyle{sysuthesis-author-year}

% 本科生参考文献的著录格式
% \usepackage[sort]{natbib}
% \bibliographystyle{sysuthesis-bachelor}

% 参考文献使用 BibLaTeX 宏包
% \usepackage[backend=biber,style=sysuthesis-numeric]{biblatex}
% \usepackage[backend=biber,style=sysuthesis-author-year]{biblatex}
% \usepackage[backend=biber,style=apa]{biblatex}
% \usepackage[backend=biber,style=mla-new]{biblatex}
% 声明 BibLaTeX 的数据库
% \addbibresource{ref/refs.bib}

% 定义所有的图片文件在 figures 子目录下
\graphicspath{{figures/}}

% 数学命令
\newcommand\dif{\mathop{}\!\mathrm{d}}  % 微分符号

% hyperref 宏包在最后调用
\usepackage{hyperref}
